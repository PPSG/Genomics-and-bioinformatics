\documentclass[a4paper,11pt]{article}

\usepackage[latin1]{inputenc}
\usepackage[T1]{fontenc}
\usepackage{bbm} %math chars
\usepackage{amsmath}
\usepackage{amsfonts}
\usepackage{indentfirst}
\usepackage{fullpage} %minimizes the default margins
\usepackage{url}
\usepackage{graphicx}
\usepackage[center,footnotesize]{caption} %options des legendes des graphes
\usepackage[section]{placeins} %place les figures d'une section avant le debut de la suivante
\usepackage{subfig} %a) b) c)
\usepackage{fancyvrb}

\title{Exercises - Week 11}
\date{}
\author{Genomics and bioinformatics}

\begin{document}
\maketitle

\section{Transcription Equilibrium}

%\begin{enumerate}
%\item 
\noindent
Solve for $m$ the differential equation
$$ \dot{m}(t) = P - \gamma\cdot m(t) \quad , $$
where $m(t)$ is the quantity of RNA in the cell at time $t$, $P$ is a constant transcription rate, 
and $\gamma$ is a constant degradation rate.
%\item Show that 
%$$ \frac{P_0}{\gamma} + \frac{1}{\omega} \sin(\omega \tau) \cos (\omega (t-\tau))$$
%is a solution of 
%$$ \dot{m}(t) = P_0 + \cos (\omega t) - \gamma \cdot m(t) \quad .$$
%\end{enumerate}

\section{Quantile Normalization}

\noindent
The expression of a few genes (g1 to g5) has been quantified and reported in the table below,
for two technical replicates R1 and R2. 

\begin{table}[tbh]
\centering
\begin{tabular}{ c | ccccc }
      &     g1   &    g2    &    g3    &    g4    &   g5 \\
      \hline
R1 &    0.8   &    1.5   &   1.7    &    2.6   &   3.9    \\
R2 &    1.1   &    2.2   &   1.6    &    2.6   &   3.8   
\end{tabular}
\end{table}

\begin{enumerate}
\item What is the median of the values in R1, and in R2?
\item Apply quantile normalization to these two samples, by hand.
\item Write a function that applies quantile normalization to any couple of vectors.
\item Test it: create two random vectors of size 2'000, apply quantile normalization to them, and
	check that after transformation they have the same median.
\end{enumerate}


\section{Linear Models}


\subsection{Continuous variable}

The overall gene expression of an individual has been quantified (by a numerologist) at different body temperatures 
and reported in the table below.

\begin{table}[tbh]
\centering
\begin{tabular}{ l | ccccccc }
Temperature &  -25  &  -10  &  -5  &  0   &  5  &  10  &  25  \\ 
Expression    &   13  &   18   &  19 &  22 &  24 & 32  &  37  
\end{tabular}
\end{table}

\noindent
We want to model the gene expression as a linear function of the temperature. The problem is to
find the ``best'' straight line across our points. A line is defined by its intercept and its slope. 
Roughly speaking, if we find a slope that is nearly zero, the temperature has no effect on gene expression, 
while a bigger slope means bigger effect.

\begin{enumerate}
\item Plot in R the expression against the temperature. Do you think a linear model can be appropriate?
Do you think that the temperature has an effect?
\item Define a linear model of the form $Y = a + b T + \epsilon$ : what does each of the terms 
correspond to in our problem (with the usual notation, close to the course's) ?
\item Apply the model to our data: each measurement of T and Y is a ``realization'' of this model, so we get 7 equations:
$$y_1 = a + b t_1 + \epsilon_1 \; ,$$
$$y_2 = a + b t_2 + \epsilon_2 \; ,$$
$$...$$
Rewrite this system as a single equation in matrix form: $Y = X\beta + \epsilon$, where this time
$Y,\epsilon \in \mathbb{R}^7$, $X \in \mathbb{R}^{7\text{x}2}$, and $\beta={a \choose b}$.
Replace $t_i$'s and $y_i$'s by their values. \\

\item We are interested in estimating from our data the intercept $a$ and the slope $b$, 
so that the the sum of all square measurement errors ($\epsilon$) is minimal. 
Let 
$$
X = (\mathbbm{1}, T) = \left( 
\begin{array}{cc}
1 & t_1 \\
1 & t_2 \\
... & ...
\end{array}
\right) .
$$
The usual estimator for $\beta$ is 
$$\hat{\beta} = (X'X)^{-1}X'Y ,$$
where $X'$ denotes $X$ transposed.\\
Use either R or Python's \textit{numpy} library to compute this vector. \\
\textit{Hints:} In R, the matrix product is \texttt{\%*\%} ; use \texttt{solve()} for the inverse.

\item Load your data (T and Y vectors) into R and do the regression the usual way, as follows: 
\begin{verbatim}
	model = lm(Y~T)
	summary(model)
\end{verbatim}	
Visualize the result:
\begin{verbatim}
	plot(T,Y)
	abline(model)
\end{verbatim}
\begin{itemize}
	\item Identify the intercept $a$, the slope $b$, the p-value for each (quality of 
		their estimation), the ``R-squared'' value (overall quality of the model (0 to 1; closer to 1 is better)).
	\item Compare with your own estimation of $\beta$. 
	\item Does the temperature significantly affect gene expression? 
	\item If one increases the body temperature by 1 degree, by how much does it increase the gene expression?
	\item Is the linear fit a good choice?
\end{itemize}

\end{enumerate}


\newpage

\subsection{Categorical variable}

The expression of a gene has been quantified several times (replicates)
in two individuals and reported in the table below. One individual has been treated by some drug, 
the other has not. We now want to model the gene expression as a linear function of the binary variable 
"Treated/Untreated".

\begin{table}[hbt]
\centering
\begin{tabular}{ l | ccccc }
                 & R1 & R2 & R3 & R4 & R5 \\
                 \hline
Untreated    & 41  & 29 & 55  & 50 & 40  \\
Treated & 43  & 35 & 60  & 53 & 42 
\end{tabular}
\end{table}

\noindent
Assign (arbitrary) numeric values to the factors: ``Treated''=1, ``Untreated''=0. 

\begin{enumerate}
\item Plot in R the expression against the treatment, similarly to the figure in the course's slides.
Do you think that the treatment has an effect?
\item Use the \texttt{boxplot} function of R to better compare graphically the two individuals.
\item Perform the same analysis as above and answer the same questions.
\end{enumerate}

\end{document}











