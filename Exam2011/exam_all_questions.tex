\documentclass[a4paper,11pt]{article}
\usepackage[latin1]{inputenc}
\usepackage[T1]{fontenc}
\usepackage{amsmath}
\usepackage{a4wide}
\usepackage{booktabs}

\title{Genomics and Bioinformatics}
\date{November 1, 2011}
\author{Examination - Week 7}
\begin{document}
\maketitle

\section*{Question 1}

Consider the following reads:
AATGCAT, GCATGCA, TGCAATG, AATGCGA

\begin{enumerate}
\item Construct the overlap graph based on the above reads, using an
  overlap size of $4$ bases.
  Draw a path that goes
  through every \emph{vertex} (Hamiltonian path), and write the
  corresponding contig. 
\item Make a list $S_4$ of all unique $4$-mers (9 elements) and the
  list $S_5$ of all (non-unique) 5-mers (12 elements). 
\item Construct the De Bruijn graph with $S_4$ as vertex set and
  $S_5$ as edge set.
\item Add one edge to make this graph Eulerian and find an Eulerian path.
  Write down the corresponding contig. 
\end{enumerate}

\section*{Question 2}
In this question you will compute the optimal global alignment between
these two sequences using the Needleman-Wunsch algorithm. 

\begin{description}
\item[Sequence 1:] \texttt{ACGTATAGGC}
\item[Sequence 2:] \texttt{ACTAAGC}
\end{description}

\begin{enumerate}
\item Sequence alignment\\
First, find the alignment that has the highest score using the
classical version of the Needleman-Wunsch algorithm. In this version a
match is worth +1 point. A mismatch is penalized -1 point. An
insertion/deletion (gap) is penalized -2 point. 
\item Affine gap penalty\\
Secondly, calculate the score of the above alignment with a gap
opening penalty of -2 and an gap extension penalty of -1. Using these
new rules, is there a different alignment with a higher score ?
\end{enumerate}

\subsection*{Hint}
The gap extension penalty is counted for every gap, the opening penalty
only for the first gap in a consecutive stretch. For example three
consecutive gaps cost $-5=-2+3*(-1)$.

For this second part, you don't need to fill a dynamic algorithm
table, just compute the score and try to improve it.


\section*{Question 3}

In this question you will build a profile HMM based on the following multiple
sequence alignment:
\begin{verbatim}
ACT
AAT
ACC
TCT
TCC
\end{verbatim}

\begin{enumerate}
\item Convert each column of this multiple alignment to nucleotide
  frequencies. Set three hidden `match' states $M_1$, $M_2$, and $M_3$
  which will emit with the frequencies of the corresponding columns in
  the alignment.
\item Set four hidden `insert' states $I_0$, $I_1$, $I_2$, and $I_3$ which
  all emit with background frequencies ($1/4$ each base).
\item Transitions are as follows: $I_n$ has transitions to itself
  (probability $20\%$) and to $M_{n+1}$ (except $I_3$), $M_n$ has transitions to
  $M_{n+1}$ (probability $80\%$) and $I_n$, except $M_3$ which only has a transition
  to $I_3$.
\item Draw the HMM diagram, write the transition matrix $\mathcal{M}$ and the
  emission matrix $\mathcal{E}$. Fill in the missing transition probabilities
  so that the matrix is consistent.
\item Complete the table below using Viterbi's algorithm and give the
  most likely sequence of hidden states 
  for the test sequence \texttt{GACAT}

  \textbf{Remark:} All
  probabilities have been multiplied by $5$ to simplify the
  calculations, the results are unchanged.  
\begin{table}[h]
  \begin{center}
    \begin{tabular}{rccccccccccc}
      \toprule
      & - & G & A & C & A & T \\
      \midrule
      $I_0$&$1$&$5/4$&$25/16$&$125/64$&$625/256$&$3125/1024$\\
      $M_1$&$0$&$0  $&$15   $&$0     $&$375/16 $&$625/32   $\\
      $I_1$&$0$&$0  $&$0    $&$75/4  $&$375/16 $&$1875/32  $\\
      $M_2$&$0$&$0  $&$0    $&$240   $&$75     $&$0        $\\
      $I_2$&$0$&$0  $&$0    $&$0     $&$300    $&$    $\\
      $M_3$&$0$&$0  $&$0    $&$0     $&$       $&$     $\\
      $I_3$&$0$&$0  $&$0    $&$0     $&$       $&$        $\\
      \bottomrule
    \end{tabular}
  \end{center}
\end{table}

\item Convert the resulting sequence of hidden states into an
  alignment containing all the sequences of the multiple alignment
  above and this last test sequence.
\end{enumerate}

\section*{Question 4}

Circadian rhythms (often referred to as the ``body clock``) are
ubiquitous in most life forms. The ``period`` family of genes is an 
essential component of this clock.

The table below provides BLAST scores of pairwise
alignments for six period family proteins from three different
species, 

\begin{enumerate}
\item {\bf dm\_per} from {\it Drosophila melanogaster}, i.e. fruit fly,
\item {\bf dr\_per2} and {\bf per3} from {\it Danio rerio}, i.e. zebra fish,
\item {\bf mm\_per1}, {\bf mm\_per2} and {\bf mm\_per3} from {\it Mus
  musculus}, i.e. mouse.
\end{enumerate}

\begin{center}
	\begin{tabular} {| l || l | l | l | l | l | l |}
	\hline
 	& dm\_per & dr\_per2 & dr\_per3 & mm\_per1 & mm\_per2 & mm\_per3 \\ \hline\hline
	dm\_per & -  & 116 & 125 & 129 & 123 & 121 \\	\hline
	dr\_per2 &	114 & - & 661 & 887 & 947& 497 \\	\hline
	dr\_per3 & 123 & 644 & - & 667 & 666 & 699 \\	\hline
	mm\_per1 & 126 & 745 & 572 & - & 640 & 80 \\	\hline
	mm\_per2 & 126 & 984 & 680 & 832 & - & 455 \\	\hline
	mm\_per3 & 124 & 499 & 709 & 537 & 488 & - \\
	\hline
	\end{tabular}
\end{center}
\vspace{0.05 cm}

\begin{enumerate}
\item Based on the table above, draw the gene tree for the six period
  proteins (dm\_per, dr\_per2, dr\_per3,  mm\_per1, mm\_per2,
  mm\_per3). 
\item Using this tree, give examples of:
  \begin{itemize} 
  \item an orthologous pair, 
  \item a paralogous pair in the same species, and 
  \item a paralogous pair in different species.
  \end{itemize}
\end{enumerate}

\end{document}
