\documentclass[a4paper,11pt]{article}

\usepackage[latin1]{inputenc}
\usepackage[T1]{fontenc}
\usepackage{bbm} %math chars
\usepackage{amsmath}
\usepackage{indentfirst}
\usepackage{fullpage} %minimizes the default margins
\usepackage{url}
\usepackage{graphicx}
\usepackage[center,footnotesize]{caption} %options des legendes des graphes
\usepackage[section]{placeins} %place les figures d'une section avant le debut de la suivante
\usepackage{subfig} %a) b) c)

\title{Series 1}
\date{September 20, 2011}
\author{Genomics and bioinformatics - Week 1}

\begin{document}
\maketitle

\section{Website of this course}
Add the course website to your bookmarks.

\url{http://moodle.epfl.ch/course/view.php?id=11181}

\section{Demo of the UCSC genome browser}
Go to the address \url{http://genome.ucsc.edu/}. Click on ``Genome browser'', select a species 
and visualize its genomic content; zoom in and out.

\section{Programming exercise}
To complete this exercise, install R and Python on your computer (see below).

This exercise, which consists in playing with a sample text file, should be completed in both languages. You must write your scripts successively in R and Python. Use as much of the documentation given below as needed to obtain the result we desire here.

Testing file: download \texttt{genes\_expression\_100.txt} on the course Moodle website. It is an output file of a typical bioinformatics program in raw text tab-delimited format - 100 lines of the form:\\

\texttt{Gene name <tab> Expression in condition 1 <tab> Expression in condition 2}\\

Using Python, then R, do: 
\begin{enumerate}
\item Read the file and extract gene names and associated numbers;
\item Compute the ratios between the two numerical columns; take $log_{2}$ of the result;
\item Compute the geometric means between the two numerical columns; take $log_{10}$ of the result;
\item Write these results in a new text file: \texttt{"GeneName <tab> $log_2$(ratio) <tab> $log_{10}$(mean)"}
\item Plot ratios vs means (use  \texttt{matplotlib} for Python).
\end{enumerate}

\clearpage
\section{R}
\subsection{Installation} 
Download R 2.13 at \url{http://stat.ethz.ch/CRAN/}\\
During the installation, when you are asked what to install, don't forget to include the help files.

If you are using Windows: during the installation, choose where you want to install your folder; then you can run it clicking on the .exe file located in the bin/ sub-folder. 

\subsection{Tutorial}
R tutorial: \url{http://cran.r-project.org/doc/manuals/R-intro.pdf}
    
If you need help:
\begin{enumerate}
\item Read the tutorial
\item Use the ? or help() R commands
\item Use Google
\item Ask us.
\end{enumerate}

\subsection{Reference documentation}
\url{http://cran.r-project.org/doc/manuals/refman.pdf}

\section{Python}

\subsection{Installing}
Download the Enthought Python from our USB keys. The packages are also available on \url{http://www.enthought.com/repo/.hidden_epd_installers} (but the server is very slow).

Open a console and type ``ipython''
(on Windows, to get a console open the ``Start'' menu and type ``cmd'' in the search field).

\subsection{Tutorial}
Python tutorial: \url{http://docs.python.org/tutorial/} read 3.1 to 4.6, 5.5 and 7.2.

\subsection{Reference documentation}
\url{http://docs.python.org/library/index.html}
\end{document}