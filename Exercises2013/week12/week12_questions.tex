\documentclass[a4paper,11pt]{article}

\usepackage[latin1]{inputenc}
\usepackage[T1]{fontenc}
\usepackage{bbm} %math chars
\usepackage{amsmath}
\usepackage{indentfirst}
\usepackage{fullpage} %minimizes the default margins
\usepackage{url}
\usepackage{graphicx}
\usepackage[center,footnotesize]{caption} %options des legendes des graphes
\usepackage[section]{placeins} %place les figures d'une section avant le debut de la suivante
\usepackage{subfig} %a) b) c)
\usepackage{fancyvrb}
\usepackage{color}
\usepackage{url}
\usepackage[colorlinks=true,linkcolor=black,citecolor=black,urlcolor=blue]{hyperref}

\begin{document}
\title{Exercises - Week 12}
\date{}
\author{Genomics and bioinformatics}
\maketitle

This series concerns the analysis of genomics data using multiple sequence alignments, DNA motifs and protein domains. More precisely, the goal of this exercise is to mimick the process of
annotating a newly discovered gene sequence, in this case the mRNA \textbf{AY239498.1} from the Genbank database. We will search homolog candidates, reconstruct their phylogeny, and analyze the potential function of the protein by identifying conserved domains and their properties.

\section{Retrieving homologs}

\begin{enumerate}
\item Search the entry AY239498.1 in the
  NCBI nucleotide database \footnote{\url{http://www.ncbi.nlm.nih.gov/nucleotide/}} and save it in
  FASTA format as \textbf{AY239498.fasta} [click on the link \textit{FASTA} at the top of the page and use the menu \textit{Send}].
\item BLAST \footnote{\url{http://blast.ncbi.nlm.nih.gov/Blast.cgi}}
  the FASTA file \textbf{AY239498.fasta} against the
  \textbf{refseq\_rna} database with parameter \textit{Max target
  sequences=50} in \textit{Algorithm parameters}. 
\item Download the resulting sequences in FASTA format (in the \textit{Descriptions} section, click on
  \textit{All$\rightarrow$Download$\rightarrow$FASTA (Complete Sequence)})
  and rename the file as \textbf{AY239498\_hom ologs.fasta}. This will be used
  later. 
\item Explore BLAST features: 
\begin{itemize}
\item Draw the corresponding tree using \textit{Distance tree of results}. Which species are present?
\item Run BLAST against the (for example) \textbf{Nucleotide collection} database. What difference does it
  make? 
\item Change the parameter values in \textit{Algorithm parameters}
  (e.g.\ \textit{Word Size}) and observe how this affects the results.
\end{itemize}
\end{enumerate}

\section{Multiple sequence alignment}

We will next use MAFFT to perform a multiple alignment of the
homologous sequences found previously. This
is a required input for phylogenetic analysis.
To improve visualization of results later, we choose to 
modify the sequence names in the FASTA file. The resulting formatted file is \textbf{AY239498\_homologs\_formated.fasta}.

\subsection{MAFFT}\label{Mafft}

\begin{enumerate}
\item Use MAFFT\footnote{\url{http://mafft.cbrc.jp/alignment/server/index.html}}
  with default parameters to perform a multiple sequence alignment of
  the file \textbf{AY239498\_homologs\_formated.fasta}. 
\item Save the results in \textit{CLUSTAL} format as \textbf{AY239498\_homologs\_formated.aln}.
\end{enumerate}


\subsection{Jalview}\label{Jalview}

\begin{enumerate}
\item Download and install Jalview:
  \url{http://www.jalview.org/download.html}
%\footnote{Some browsers
%    may lead to a broken installation file; in this case one should try to use the command
%    \emph{wget}.}.
\item Launch Jalview and visualise the alignment
  (\textit{File$\rightarrow$Input Alignment$\rightarrow$From File}). 
\item Test various features of Jalview:
\begin{itemize}
\item Color nucleotides (menu \textit{Colour$\rightarrow$Nucleotides}).
\item Display a global view (menu \textit{View$\rightarrow$Overview
  window}) and try to identify regions with the best alignment. 
\item Display the logo of the consensus sequence (right-click on the consensus and select \textit{Show Logo}).
\item Edit the alignment (remove divergent or short sequences, etc.)
\end{itemize}
\item Select the block from position \textbf{1108} to \textbf{2461},
  in menu \textit{Edit} do \textit{Copy/Paste to new
alignment}, and export the sequences in a FASTA file \textbf{conserved\_block.fasta} (use
\textit{save as}).
\end{enumerate}

\section{Inferring the evolutionary tree}
\subsection{RAxML}
\begin{enumerate}
\item Run \textbf{RAxML}\footnote{\url{http://phylobench.vital-it.ch/raxml-bb/}} with
  the multiple alignment \textbf{AY239498\_homologs\_formated.aln} to
  infer the phylogenetic tree. \textit{If it takes too
  long, go to the next section}.
\item Save the MRE consensus tree in a file named
  \textbf{AY239498\_MRE.tree}.
\end{enumerate}


\subsection{Dendroscope}
\begin{enumerate} 
%http://ab.inf.uni-tuebingen.de/data/software/dendroscope3/download/welcome.html
\item Install Dendroscope\footnote{\url{http://ab.inf.uni-tuebingen.de/software/dendroscope/}}
\item Visualise the consensus tree
  \textbf{AY239498\_MRE.tree} obtained
  in the previous section.
\item Explore the various visualization modes, re-root the tree.
\end{enumerate}

\section{Identifying protein domains}

The conserved block identified in the multiple sequence alignment
probably contains a functionally important domain of the protein. We will search
databases for proteins containing the same sequence block and see
which characterized functional domains they contain.

\subsection{BLASTX}

\begin{enumerate}
\item Run BLASTX\footnote{\url{http://blast.ncbi.nlm.nih.gov/}} with
  the conserved sequence block saved in section \ref{Jalview}
  (\textbf{conserved\_block. fasta})
  against the \textbf{swissprot} database with default parameters. 
\item Save the top protein hits in FASTA format for the next step (\textbf{protein\_sequences.fasta}).
\end{enumerate}

\subsection{Prosite and InterPro searches}

\begin{enumerate}
\item Search the BLASTX result \textbf{protein\_sequences.fasta} on Prosite\footnote{\url{http://prosite.expasy.org/scanprosite/}}.
\item Search the first protein sequence
  (\textbf{P22555}) on
  InterProScan\footnote{\url{http://www.ebi.ac.uk/Tools/pfa/iprscan/}}.
\item Which domain is found by these tools?
\item Are there domains described only in one source? Note the differences between the descriptors of the same domains.
\end{enumerate}

\section{DNA motif discovery}

We have identified our sequence \textbf{AY239498.1} as coding for a
\textbf{c-Myc} homolog and containing a \textbf{bHLH} DNA-binding
domain. To determine its
targets, we might perform a ChIP-seq experiment.
Today we will use the data from a similar experiment on the
Human c-Myc
(from the ENCODE project \url{http://www.genome.ucsc.edu/ENCODE/}). 
Sequencing data have been mapped to the Human genome and, after peak calling,
a list of chromosomal coordinates containing 
the genomic regions bound by c-Myc are obtained. We will use them to identify
the motif specifically recognized by c-Myc's bHLH domain.

\subsection{Retrieve sequences by their genome coordinates (with UCSC)}

\begin{enumerate}
\item Download the BED file \textbf{posPeakMyc.bed} and go to the UCSC genome
  browser\footnote{\url{http://genome.ucsc.edu/}}, click on \textit{Genome}
  (top left) and select the Human genome assembly hg19. Click on
  \textit{manage custom tracks}, then on \textit{add custom
    tracks} and load the bed file.  
\item Click on \emph{go to table browser} and select \textit{sequence} in
  the list of output format. Enter \textbf{posPeakMyc.fasta} in field
  output file. Click on \emph{get output}. 
\item In this page, select the \textit{All upper case} option and
  click on \textit{get sequence} to save the sequences into file
  \textbf{posPeakMyc.fasta}.
\item In the result file, you can check that the first line looks like:
\end{enumerate}
\begin{verbatim} 
>hg19_ct_UserTrack_3545_(null) range=chr1:8939139-8939538 5'pad=0 
3'pad=0 strand=+ repeatMasking=none
\end{verbatim}

\newpage
\subsection{ Mask low complexity and repeated elements}

Motif search can be hindered by repetitive sequences included in the
binding regions. We will use a tool to identify
and remove potential repetitive sequences.

\begin{enumerate}
\item Go to repeatMasker\footnote{\url{http://www.repeatmasker.org/}}. The
  interest of this tool is to mask regions of low complexity to
  facilitate the identification of relevant motifs. 
\item Choose \textit{RepeatMasking} service.
\item Select your file \textbf{posPeakMyc.fasta} and submit the job with default parameters.
\item Check which elements have been masked in \textit{xxx.out.html} and save the masked file \textit{xxx.masked}. 
\item Which difference can you see with the input file ?
\end{enumerate}

\subsection{Motif discovery with MEME-ChIP}

\begin{enumerate}
\item Go to
  MEME-ChIP\footnote{\url{http://meme.sdsc.edu/meme/cgi-bin/meme-chip.cgi}},
  and fill the form with the masked sequences \textbf{posPeakMyc\_masked .fasta} previously
  obtained.

Note: In MEME options, we recommend to choose one occurrence per sequence, a maximum width of 15, and a maximum number of motif of 5.
\item Have a look at the MEME and TOMTOM html results.
\item Which transcription factor corresponds to the main motif? 
\end{enumerate}

\end{document}
