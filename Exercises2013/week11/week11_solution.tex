\documentclass[a4paper,11pt]{article}

\usepackage[latin1]{inputenc}
\usepackage[T1]{fontenc}
\usepackage{bbm} %math chars
\usepackage{amsmath}
\usepackage{indentfirst}
\usepackage{fullpage} %minimizes the default margins
\usepackage{url}
\usepackage{graphicx}
\usepackage[center,footnotesize]{caption} %options des legendes des graphes
\usepackage[section]{placeins} %place les figures d'une section avant le debut de la suivante
\usepackage{subfig} %a) b) c)

\title{Exercises - Week 11}
\date{}
\author{Genomics and bioinformatics}

\begin{document}
\maketitle

\section{Transcription Equilibrium}

\noindent
$\frac{dm(t)}{dt} = P - \gamma m(t)
\iff \frac{1}{P-\gamma m(t)} \, dm(t) = dt
\iff \int_{m(0)}^{m(t)} \frac{1}{P-\gamma u} \, du = \int_0^t \, ds \\
\iff \frac{-1}{\gamma} (\log(P-\gamma m(t)) - \log(P-\gamma m(0))) = t - 0
\iff \log(\frac{P-\gamma m(t)}{P-\gamma m(0)}) = -\gamma t \\
\iff \frac{P-\gamma m(t)}{P-\gamma m(0)} = e^{-\gamma t}
\iff m(t) = \frac{1}{-\gamma} \cdot (e^{-\gamma t} (P-\gamma m(0)) - P ) 
= -\frac{P}{\gamma} \, e^{-\gamma t} + m(0)\, e^{-\gamma t} + \frac{P}{\gamma}
$ \\\\
which tends to $\frac{P}{\gamma}$ when $t$ tends to infinity ($e^{-\gamma t}$ tends to zero).

\section{Quantile Normalization}

\begin{enumerate}
\item R1: 1.7 - R2: 2.2.
\item Record the initial order in R1 (g1-g2-g3-g4-g5) and R2 (g1-g3-g2-g4-g5).
Sort their values:
\begin{tabular}{ cccccc }
R1 &    0.8   &    1.5   &   1.7    &    2.6   &   3.9    \\
R2 &    1.1   &    1.6   &   2.2    &    2.6   &   3.8    \\
\end{tabular}

Calculate the mean of each pair:

\begin{tabular}{ cccccc }
Avg &   0.95  &    1.55  &   1.95  &    2.6   &   3.85   \\
\end{tabular}

Replace R1 and R2 by these same values:

\begin{tabular}{ cccccc }
R1' &   0.95  &    1.55  &   1.95  &    2.6   &   3.85    \\
R2' &   0.95  &    1.55  &   1.95  &    2.6   &   3.85    \\
\end{tabular}

Reorder as it was initialy:

\begin{tabular}{ cccccc }
      &     g1   &    g2    &    g3    &    g4    &   g5 \\
R1 &   0.95  &    1.55  &   1.95  &    2.6   &   3.85    \\
R2 &   0.95  &    1.95  &   1.55  &    2.6   &   3.85    \\
\end{tabular}

Now the median and all other quantiles are the same in R1 and R2.
 
\item 4. 5. : see \texttt{week11\_solution.R} .
\end{enumerate}


\section{Linear Models}

\subsection{Continuous variable}

\begin{enumerate}
\item At first view points are roughly aligned and expression seems to increase with temperature. To plot it:
\begin{verbatim}
T = c(-25,-10,-5,0,5,10,25)
Y = c(13,18,19,22,24,32,37)
plot(T, Y, xlim=c(-30,30), ylim=c(0,50))
\end{verbatim}

\item 
$Y$ is the response: a random variable generating the gene expression values, assumed normally distributed. \\
$T$ is the factor: the temperature. \\
$\text{a}$ is the intercept: the point where the line crosses the vertical axis (unknown). \\
$b$ is the slope of the line (unknown). \\
$\epsilon$ is the measurement error (a normally distributed random variable).

\item 
$$13 = a -25 b + \epsilon_1 \; ,$$
$$18 = a -10 b + \epsilon_2 \; ,$$
$$...$$
In matrix form:
$$
\left(
\begin{array}{l}
13 \\ 18 \\ 19 \\ 22 \\ 24 \\ 32 \\ 37 
\end{array}
\right)
=
\left(
\begin{array}{ccccccc}
1 & -25 \\ 
1 & -10 \\ 
1 & -5 \\ 
1 & 0 \\ 
1 & 5 \\ 
1 & 10 \\ 
1 & 25 
\end{array}
\right)
\left(
\begin{array}{c}
a \\ b
\end{array}
\right)
+
\left(
\begin{array}{l}
\epsilon_1 \\ \epsilon_2 \\ \epsilon_3 \\ \epsilon_4 \\ \epsilon_5 \\ \epsilon_6 \\ \epsilon_7 
\end{array}
\right)
$$

\item
\begin{verbatim}

# "%*%" is the matrix product
# solve() gives the inverse
# Vectors are "vertical" by default

X = cbind(rep(1,7),T)
beta = solve(t(X) %*% X ) %*% (t(X) %*% Y)
\end{verbatim}
One finds $a=23.57$ and $b=0.51$.

\item The output is the following:
\begin{verbatim}
> summary(lm(Y~T))

Call:
lm(formula = Y ~ T)

Residuals:
      1       2       3       4       5       6       7
 2.1786 -0.4714 -2.0214 -1.5714 -2.1214  3.3286  0.6786

Coefficients:
            Estimate Std. Error t value Pr(>|t|)
(Intercept) 23.57143    0.88744  26.561 1.41e-06 ***
T            0.51000    0.06062   8.413 0.000389 ***
---
Signif. codes:  0 '***' 0.001 '**' 0.01 '*' 0.05 . 0.1 ' ' 1

Residual standard error: 2.348 on 5 degrees of freedom
Multiple R-squared:  0.934,	Adjusted R-squared:  0.9208
F-statistic: 70.77 on 1 and 5 DF,  p-value: 0.0003891
\end{verbatim}

The Intercept-Estimate is the $a$, 23.57; the T-Estimate is the $b$, 0.51. 
We recognize the numbers that we found with the given formula for $\beta$.
The "Pr(>|t|)" column contains the p-values. The one for $b$ is low enough to say that the temperature 
has a significant effect on gene expression. For instance, an increase of 1 in temperature induces an increase 
of 0.51 in expression. The R-squared is very close to 1, which means that the points lie close
to the fitted line, thus the linear model is probably appropriate.

\end{enumerate}

\subsection{Categorical variable}

\begin{enumerate}
\item 
\begin{verbatim}

untreated = c(41,29,55,50,40)
treated = c(43,35,60,53,42)
T = c( rep(0,5), rep(1,5) )
Y = c( untreated, treated )
plot(T, Y, xlim=c(-0.5,1.5), ylim=c(20,70))
\end{verbatim}
We notice a systematic increase in the treated sample, but the variance is big and
it is hard to decide if there really is an effect.

\item
\begin{verbatim}

boxplot(Y~T, names=c("Untreated","Treated"))
\end{verbatim}

\item The system can be written
$$41 = a + 0\cdot b + \epsilon_1 \; ,$$
$$29 = a + 0\cdot b + \epsilon_2 \; ,$$
$$...$$
$$43 = a + 1\cdot b + \epsilon_1 \; ,$$
$$35 = a + 1\cdot b + \epsilon_2 \; ,$$
$$...$$
or
$$
\left(
\begin{array}{c}
41 \\ 29 \\ 55 \\ 50 \\ 40 \\ 43 \\ 35 \\ 60 \\ 53 \\ 42
\end{array}
\right)
=
\left(
\begin{array}{cc}
1 & 0 \\ 1 & 0 \\ 1& 0 \\ 1 & 0 \\ 1 & 0 \\ 1 & 1 \\ 1 & 1 \\ 1 & 1 \\ 1 & 1 \\ 1 & 1
\end{array}
\right)
\left(
\begin{array}{c}
a \\ b
\end{array}
\right)
+
\left(
\begin{array}{c}
\epsilon_1 \\ \epsilon_2 \\ \epsilon_3 \\ \epsilon_4 \\ \epsilon_5 \\ \epsilon_6 \\ \epsilon_7 \\ \epsilon_8 \\ \epsilon_9 \\ \epsilon_{10} 
\end{array}
\right)
$$

\item
\begin{verbatim}

X = cbind(rep(1,10),T)
beta = solve(t(X) %*% X ) %*% (t(X) %*% Y)
\end{verbatim}
One finds $a=43$ and $b=3.6$.

\item The output is the following:
\begin{verbatim}
> summary(lm(Y~T))

Call:
lm(formula = Y ~ T)

Residuals:
   Min     1Q Median     3Q    Max
-14.00  -4.35  -2.50   6.85  13.40

Coefficients:
            Estimate Std. Error t value Pr(>|t|)
(Intercept)   43.000      4.447   9.668 1.09e-05 ***
T              3.600      6.290   0.572    0.583
---
Signif. codes:  0 '***' 0.001 '**' 0.01 '*' 0.05 '.' 0.1 ' ' 1

Residual standard error: 9.945 on 8 degrees of freedom
Multiple R-squared:  0.03934,	Adjusted R-squared:  -0.08074
F-statistic: 0.3276 on 1 and 8 DF,  p-value: 0.5828
\end{verbatim}
This time the estimate for $b$ has a p-value of about 0.6, which means one cannot
trust the result at all. Overall the fit is terrible with an R-squared of 0.04 and a bad F-statistic.
However, the estimate for $b$, 3.6, is the systematic increase that we noticed earlier.

\end{enumerate}

\end{document}











