\documentclass[a4paper,11pt]{article}

\usepackage[latin1]{inputenc}
\usepackage[T1]{fontenc}
\usepackage{bbm} %math chars
\usepackage{amsmath}
\usepackage{indentfirst}
\usepackage{fullpage} %minimizes the default margins
\usepackage{url}
\usepackage{graphicx}
\usepackage[center,footnotesize]{caption} %options des legendes des graphes
\usepackage[section]{placeins} %place les figures d'une section avant le debut de la suivante
\usepackage{subfig} %a) b) c)

\title{Series 2}
\date{September 27, 2011}
\author{Genomics and bioinformatics - Week 2}

\begin{document}
\maketitle

\section{Description}
In today's session you will retrieve publicly available genome sequence and annotation data 
\indent for a particular species and use it to extract some biological information about that species.

\section{Downloading the genome data}
From the UCSC Genome Browser select the latest assembly of the $Mus$ $musculus$ genome.

Scroll down to ``Assembly details'' and click on the ``Downloads'' link.

From ``Data set by chromosome'' download the \texttt{.fa.gz} file for any chromosome.

From ``Annotation database'' download the \texttt{knownGene.txt.gz} and \texttt{kgAlias.txt.gz} files.

\section{Programming exercise}

\subsection{Using Python}
\begin{enumerate}
\item Read the \texttt{.fa} file and extract the chromosome sequence
\item Determine the length of the sequence
\item Calculate the number of As, Gs, Cs and Ts in the sequence
\item Compute the GC-content of the chromosome
\item Plot GC content vs chromosome length (use  \texttt{matplotlib})
\end{enumerate}

\Large{\emph\bf Questions}
\begin{enumerate}
\normalsize\item What does the GC-content tell us about a genome? 
\item Using data for all chromosomes, calculate the average GC-content of the mouse genome. Does this compare to the value mentioned in your lecture slides?
\end{enumerate}

\subsection{Using R}
\begin{enumerate}
\normalsize\item Read the \texttt{knownGene.txt} and \texttt{kgAlias.txt}  files
\item Extract the annotation corresponding to your chromosome (\texttt{chrom} column) from \texttt{knownGene.txt}
\item Extract a list all the exons in the chromosome (\texttt{exonStarts}, \texttt{exonEnds} columns)
\item Compute exon sizes. Create a table as shown below,

\scriptsize geneID exonNum exonStart exonEnd exonSize

\scriptsize uc007aet.1	1	3195984	3197398	1415

\scriptsize uc007aet.1	2	3203519	3205713	2195

\scriptsize uc007aeu.1	1	3204562	3207049	2488

\scriptsize uc007aeu.1	2	3411782	3411982	201

\scriptsize uc007aeu.1	3	3660632	3661579	948

\normalsize\item Plot the exon size distribution
\end{enumerate}

\normalsize\underline{Note}

\normalsize The \texttt{knownGene.txt} and \texttt{kgAlias.txt} files are both tab-delimited files.

Column headers for the \texttt{knownGene.txt} file are,

\scriptsize\texttt {geneID	chrom	strand	txStart	txEnd	cdsStart	cdsEnd	exonCount	exonStarts	exonEnds	proteinID	alignID}

\normalsize {Column headers for the \texttt{kgAlias.txt} file are,}

\scriptsize\texttt {geneID	geneName}\\

\Large {\emph\bf Questions}
\begin{enumerate}
\normalsize\item List names of genes with,

 a) the longest exon, b) the shortest exon and c) most number of exons. 

\item List all the intronless genes in the chromosome. 
\item What can you tell about the exon size distribution across different mouse chromosomes?
\end{enumerate}

\normalsize\underline{Note}

\indent\normalsize Cross \texttt{knownGene.txt} and \texttt{kgAlias.txt} to obtain \texttt{geneNames} corresponding to each \texttt{geneID}. 

\scriptsize\texttt  geneID geneName

\scriptsize\texttt  uc007aet.1	AK135172, mKIAA1889, uc007aet.1

\scriptsize\texttt  uc007aeu.1	NM\_001011874, NP\_001011874, Q5GH67, XKR4\_MOUSE, Xkr4, Xrg4, uc007aeu.1

\subsection{Reference documentation}
\normalsize For R - \url{http://cran.r-project.org/doc/manuals/refman.pdf}

For Python - \url{http://docs.python.org/tutorial/}\\

\emph{If you need help:}
\begin{enumerate}
\item Go through last week's exercise session for examples
\item Use Google
\item Use the ? or help() with R commands
\item Ask us
\end{enumerate}

\end{document}