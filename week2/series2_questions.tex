\documentclass[a4paper,11pt]{article}
\usepackage[latin1]{inputenc}
\usepackage[T1]{fontenc}
\usepackage{bbm} %math chars
\usepackage{amsmath}
\usepackage{indentfirst}
\usepackage{fullpage} %minimizes the default margins
\usepackage{url}
\usepackage{graphicx}
\usepackage[center,footnotesize]{caption} %options des legendes des graphes
\usepackage[section]{placeins} %place les figures d'une section avant le debut de la suivante
\usepackage{subfig} %a) b) c)
\title{Series 2}
\date{}
\author{Genomics and bioinformatics - Week 3 - September 27, 2012}
\begin{document}
\maketitle

\section{Overlap graphs}
Here are five "reads'': TCTATA, GGTCTA, TATCTA, TCTAGC, TATATC. They originate from a single, longer contig.
\begin{enumerate}
\item Try to reassemble the contig by hand, looking at the overlaps between the reads.
\item Build an (Hamiltonian) overlaph graph with these reads, and draw the longest path.
From there, reconstruct the contig.
\item Take $l=4$ and build the corresponding ``de Bruijn'' graph. Find an Eulerian cycle, and reconstruct the contig. \\
\textit{Note}: The ``de Bruijn'' graph must be Eulerian, thus some edges of the graph will have to be doubled.
Try to understand which ones you must double (look at the overlaps between reads), and think about a
criterion to do it automatically.
\end{enumerate}

\section{Getting data with UCSC}
\subsection{Visualizing genome data}
\begin{enumerate}
\item Go to the UCSC Genome Browser and select the mouse genome.
\item Visualize the most recent assembly (mm10) of chromosome 18.
\item Scroll down to ``Mapping and Sequencing Tracks'' and load the GC percent track.
\end{enumerate}

\subsection{Downloading genome data}
Copy the ``sequence'' file \texttt{chr18.fa} and ``annotation'' file \texttt{chr18.gtf} for the mouse chr18 from the USB keys provided by us. \\\\
\textit{Note}: The \texttt{.fa} file for the mouse can be downloaded from the UCSC Downloads page:\\
\url{http://hgdownload.cse.ucsc.edu/goldenPath/mm10/chromosomes/},\\
and the \texttt{.gtf} file for the whole mouse genome can be downloaded from ENSEMBL:\\
\url{ftp://ftp.ensembl.org/pub/release-64/gtf/mus_musculus/} .


\section{Manipulating data with Python}
\begin{enumerate}
\item Load the \texttt{.fa} file for chr18 and extract the sequence.
\item Determine the length of the sequence (see \texttt{len}).
\item Calculate the number of As, Gs, Cs and Ts in the sequence.
\item Compute the GC content of the chromosome.
\item Plot GC content along mouse chr18 using an appropriate window (bin) sizes (use \texttt{matplotlib}).
\item Write the start and end coordinates of each bin and it's corresponding GC content to a file, as follows:
\texttt {binStart <tab> binEnd <tab> GC\_content}
\end{enumerate}
\textit{Note}: if you feel confident, this is a good occasion to save time trying the Biopython library: \\
\texttt{from Bio import seqIO} \# then use seqIO.read()\\
\texttt{from Bio.SeqUtils import GC} \# then use GC() \\
\textit{Note}: list methods such as \texttt{count}, \texttt{append}, etc. may be useful.



\section{Manipulating data with R}
\subsection{Exons}
The \texttt{.gtf} file is a tab-delimited file, with the following column headers: \\
\texttt {chromosome source feature start end score strand frame attributes} .
\begin{enumerate}
\item Load the \texttt{.gtf} file for chr18 in R.
\item Extract the rows corresponding to exons from the \texttt{feature} column to a new table.
\item Compute exon sizes, and attach them to the table in a new column ``\texttt{exonSize}''.
\item Plot the exon size distribution for chr 18.
\end{enumerate}

\subsection{GC content}
\begin{enumerate}
\item Load the file (table) generated by your python script into R.
\item Recreate the GC content plot for chr18 in R.
\end{enumerate}

\subsection{Some new useful features for this exercise}
\begin{itemize}
\item \texttt{which, length, max, hist}
\item Loops and conditions: \texttt{for, if, in}
\item Conversions: \texttt{as.vector, as.numeric, as.data.frame, as.factor, float, int,}...
\end{itemize}

\end{document}
