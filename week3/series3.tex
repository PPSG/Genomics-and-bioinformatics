
\documentclass[a4paper,11pt]{article}

\usepackage[latin1]{inputenc}
\usepackage[T1]{fontenc}
%\usepackage[francais]{babel}
\usepackage{bbm} %math chars
\usepackage{amsmath}
\usepackage{indentfirst}
\usepackage{fullpage} %minimizes the default margins
\usepackage{url}
\usepackage{graphicx}
\usepackage[center,footnotesize]{caption} %options des legendes des graphes
\usepackage[section]{placeins} %place les figures d'une section avant le debut de la suivante
\usepackage{subfig} %a) b) c)

\title{Series 1}
\date{September 20, 2011}
\author{Genomics and bioinformatics - Week 1}


\begin{document}
\maketitle


\section{Aim}
The final aim of the programming exercises is you to be able to
download some interesting data files from UCSC, format them
(which is usually the hardest part), extract relevant data from them
and apply some analysis that you will learn during this course.

\section{Exercise 1}

\section{Exercise 2}

\newpage

\section{Python tricks}
\begin{enumerate}
\item You don't need to use the indexes of a vector to iterate over its elements.
\item Use ';' to put several statements on the same line. Else it is not needed.
\item dir(object) tells you all the existing methods for this object. If you are using
Ipython, you can type object. and press Tab to display it all.
\end{enumerate}

\section{R tricks}

\end{document}